A large literature has shown the adverse effects of job loss on workers. \citet{JLS1993} show that workers involved in a mass layoff lose about 25 percent of their earnings over the next six years. Additionally, \citet{Stevens1997} shows that much of this effect can be explained by  individuals losing subsequent jobs, effectively compounding the adverse effects. However, recent literature has shown that while the mean effect of job loss is quite negative, the dispersion of outcomes for displaced workers is large; \citet{FHM2012} find that a worker at the 10th percentile of outcomes experiences a 10 percent decline in earnings, while a worker at the 90th percentile sees a 14 percent increase in earnings following a mass layoff. These results suggest that some workers actually benefit from being laid off, and one potential channel to explain this earnings growth is increased human capital accumulation following the separation.

There is also a growing literature documenting worker labor force exit following adverse labor demand shocks. \citet{ADH2013} show that workers in areas experiencing lower labor demand because of increased trade competition drop out of the labor force at higher rates than workers in areas not exposed to this competition. \citet{FGS2015} find that following a mass layoff, about half of the adjustment in labor force size is due to non-participation, but what workers do upon exit is not examined.  \citet{Yagan2016} also shows that non-participation is particularly important in the Great Recession.

In addition to findings on worker displacement, there is strong evidence that postsecondary enrollments are counter-cyclical. \citet{BF1995} find that unemployment rate increases of one percentage point lead to enrollment increases of four percentage points, and similar recent work uses similar designs \citep{HO2013, Nutting2008, clark2011}. Additionally, recent evidence shows that enrollment in the two-year and four-year sectors increased considerably due to the Great Recession \citep{BD2012}, but there is much less evidence on whether the content of what people study is affected by downturns. An exception is \citet{Nutting2008}, who finds that career technical enrollment is more responsive to labor market conditions than academic enrollment at one large public university.\footnote{A growing number of papers using natural experiments focus on educational impacts of natural resource booms \citep{Basso2016, BMS2005, MCL2015,CN2015},\footnote{\citet{Basso2016} also examines the effect of the bust following an oil boom.} though most consider the high school dropout margin as opposed to the college enrollment decision.}

Recent work  also examines the skill upgrading of displaced workers. \citet{BT2015} find that longer unemployment insurance duration and more generous unemployment insurance policies increase the likelihood of individuals to enroll in postsecondary education. \citet{JLS2005b, JLS2005a} use administrative data from Washington and find that workers who enroll in schooling following a job loss have increased earnings, even for older workers. 

There is also growing interest in how postsecondary institutions themselves act as economic agents, and whether they respond to local changes in labor demand. This interest is partially related to the rise of  two-year for-profit colleges, leading to concern about competition between the public and the private sector \citep{DGK2012, Cellini2009, Cellini2010}. \citet{Xia2016} shows that for-profit schools compete with community colleges over specific programmatic offerings, though on the whole they seem more responsive to incentives from financial aid availability than the local demand for skills \citep{Cellini2010b}. \citet{Xia2016} also finds that community colleges are rigid in their course offerings and class  sizes.
