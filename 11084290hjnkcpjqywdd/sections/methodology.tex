
Our goal is to measure the effect of local economic shocks on educational decisions of individuals. To avoid issues of endogeneity and measurement error inherent in using the unemployment rate, we use  mass layoff counts as a measure of local labor demand shocks.\footnote{We discuss the drawbacks of regressing local outcomes on the unemployment rate in more detail in Appendix \ref{sec:appunemp}. For completeness,  Appendix \ref{sec:appunemp} also shows results using this approach.}  Mass layoffs measure acute local shocks to the labor demand, with fewer concerns about measurement error inherent in small area estimation. Moreover, mass layoffs are of independent interest as a concrete measure of the number of jobs lost in a particular type of displacement.\footnote{A commonly used instrument for the unemployment rate to isolate labor demand shocks is the shift-share or ``Bartik'' instrument \citep{SW2011, BH2000}, which leverage pre-existing area-specific industry structure and changes in industry outcomes at the national level. There are a number of reasons why such a method is likely inadequate in the context we study. First, these insstruments do not usually give intuition on the  size of a shock relative to the local labor force. More importantly, the shift-share instruments are better suited to identifying long-run structural shocks, as opposed to the transitory, acute impact of layoffs that we study in this paper. Moreover, as we discuss in \citet{FGS2015}, Bartik demand measures combine local industry structure with trends in national product and labor demand. Education may respond to both local and national business cycle conditions, but here we focus on local measures, so it is not desirable to combine local and national conditions  in a single measure. }

We estimate the effect of a large mass layoff event on enrollment in postsecondary education. In particular, we estimate an elasticity, and allow mass layoffs to impact enrollment in more than one period, since it takes workers time to adjust. To do so, we estimate the following equation:

\begin{equation}\label{eqn:main}
y_{ct} = \sum_{i=1}^I \beta_i m_{c,t-i} + \Theta X_{ct} + \gamma_c + \eta_t +\xi_c*t+ \epsilon_{ct}
\end{equation}

Where $y_{ct}$ is the one of our outcomes of interest, either enrollment or awards, while $m_{ct}$ is the log of the number of workers directly affected by mass layoff events in commuting zone $c$ in year $t$.\footnote{We use commuting zones as our geographic unit of observation because a worker is likely to commute to school within this area.} Since both the outcome and the key regressor are expressed as logs, our estimate of $\beta_i$ can be interpreted as an elasticity.

In the matrix $X_{ct}$ we include a set of time-varying measures of local characteristics. We include commuting zone fixed effects, $\gamma_c$, to account for systematic, time-invariant differences between commuting zones, and  year fixed effects, $\eta_t$, to control for national trends.  All regressions are weighted by the local labor market's lagged total population. Finally, to address the fact that mass layoffs may be correlated within a commuting zone over time, we cluster our standard errors at the commuting zone level.

A crucial element of our main estimating equation is the inclusion of market-specific linear trends $\xi_c*t$. These labor market specific trends  effectively identify the key coefficient $\beta_i$ off deviations from those trends. It is important to account for these trends for a number of reasons. First, we want to control for long-run economic conditions which may also be correlated with funding of community colleges, and other local policies. Second, the demographic composition of certain areas may be changing  over our time period, making the area's population increasingly more or less likely to attend postsecondary institutions. Not controlling for these differential demographic changes would  confound the effect of mass layoffs with demographic changes. Finally, because mass layoffs are concentrated in certain areas of the country, including commuting zone-level trends  identifies the effect off shocks that are deviations from the common trend, as opposed to large shocks per se.\footnote{As an example, consider a commuting zone in which labor demand is decreasing over time, and so there are an increasing number of layoffs every year. Including local trends acknowledges this commuting-zone-specific growth, and the identification comes from layoff events that depart from the trend.}  Given the importance of including local trends, we include them in all our analyses, and present results without linear trends in Appendix Table \ref{tab:firstmain}.

One potential limitation of our approach is that we estimate equation \ref{eqn:main} at an aggregate level, as opposed to using individual-level data on job loss. Thus we are unable to assert that the workers losing their jobs are exactly the individuals enrolling in postsecondary institutions: it may be the case that other workers are responding to slack labor market conditions and choosing to enroll in school. Additionally, there is a possibility that some students are affected because their parents lost their jobs, and choose to enroll in community college as a less expensive alternative to a four-year college \citep{hilger2014}. For this to be a large portion of the effect, we would expect a commensurate decline in four-year school enrollment. We do not find such an effect when estimating equation \ref{eqn:main}, which limits our concern about this issue.

As shown in much of the prior literature, there is a dramatic migration response to local economic downturns \citep{blanchard1992regional}. While we do not explicitly estimate it here, in related work we find a migration response to mass layoffs as well \citep{FGS2015}. If some students are being induced to move across labor markets following economic shocks, this will reduce our estimates of $\beta_i$. In that sense, our estimates are a lower bound for the the effect of mass layoffs on the educational enrollment of students who remain in the labor market. 