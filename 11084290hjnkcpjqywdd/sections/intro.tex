 The dynamics of worker adjustment following job loss have gained renewed interest in light of the recent Great Recession and  declines in labor force participation \citep{wh2016}. Some workers  regain employment quickly, but many exit the  labor force. However, it remains an open question what workers do upon exiting the labor force, especially as they take an increasingly long time to re-enter the ranks of the employed.\footnote{In previous work, we have considered cross-county migration flows, enrollment in public benefit programs, and non-participation \citep{FGS2015}.} Workers who exit the labor force may experience significant depreciation in human capital, which has permanent productivity and wage consequences \citep{AES1999,GdG2009}. However, some workers exit the labor force to upgrade their skills.  Thus, it is crucial to understand to what extent workers enter postsecondary institutions in bad economic times, as well as what fields they study.
 
There is considerable research documenting the relationship between job loss and postsecondary enrollment. However, the literature tends to measure either individual responses to personal job loss \citep[e.g.][]{BT2015} or aggregate effects of large macroeconomic shocks \citep[e.g.][]{BD2012}. Much less is known about the effect at the local labor market level, where the contraction of specific industries or downsizing of large employers may have outsized ripple effects. It is particularly important to understand these dynamics across the entire local labor market since some workers not directly affected by job loss may also respond to an adverse economic climate by enrolling in school.

In this paper, we consider postsecondary education as a channel for worker adjustment following local economic shocks. Community colleges are the primary engines of worker retraining, especially for older and non-traditional students. Because these are the workers who are most likely affected by local economic downturns, understanding the role of community colleges in the adjustment of local labor markets is particularly important. We consider  enrollment outcomes as well as the receipt of degrees and certificates, and we distinguish between two-year associate's degrees and other certificates and diplomas. Additionally, we group awards by field of study and duration, focusing in particular on how degree completion in career-technical education fields responds to labor market shocks. We also pay close attention to fields with rapidly rising or declining labor demand, such as  health, information technology, and manufacturing. 

This paper makes several contributions to the literature. In contrast to previous papers in the literature, we use mass layoffs as a  measure of adverse labor demand changes. Mass layoffs are  a large, acute shock to employment for workers, and not open to many of the methodological limitations common to studies of this kind.\footnote{As we outline in a later section, many similar studies use the unemployment rate as the measure of local demand shocks. This has several drawbacks, not least of which is the endogeneity of the unemployment rate to exits from the labor force such as educational enrollment.} Second, our analysis  spans the entire country, while other papers have considered  specific industries, areas, or workers. Third, we focus primarily on the effects at the two-year college level, in the spirit of the influential  analysis by \citet{BF1995}. While we replicate their finding on the counter-cyclicality of enrollment with respect to local labor demand shocks, we also separately estimate effects for completion in different types of degree and certificate programs. Finally, our data allow us to test for differences in response during the Great Recession, when labor  demand shocks were particularly acute. 

We find that following mass layoff events, there is a marked increase in community college enrollment. Our main results suggest that for every 100 workers laid off, enrollment increases by three students within the next three years. This enrollment channel accounts for approximately half of the size of the labor force non-participation following mass layoff events, which we document in \citet{FGS2015}. Additionally, we find that for 100 workers involved in a mass layoff, approximately two workers end up receiving a degree or certificate. When we examine different types of awards separately, we find that the bulk of this effect is concentrated among certificates, as opposed to associate's degree programs. 

We also find that there is heterogeneity in the extent to which workers complete different programs following layoff events. There are much larger responses in career-technical fields than in academic fields focused on transfer to four-year colleges. Within career-technical fields, we find particularly large effects for associate's degrees in construction and manufacturing, as well as certificate programs in allied health.\footnote{Examples of allied health programs include medical and nursing assistants, dental hygienists and assistants, emergency medical technicians, and radiologic technologists.} Overall, we find a positive though weak correlation between a particular program's completion response to layoff events and its estimated earnings return, which we take as suggestive evidence that workers leaving the labor force to retrain enroll in high-return fields. 

%In future work, we also want to look at how TAA and WIA funding levels increase or dampen these enrollment and receipt effects, and how this differs based on the size of geographic location of the area where the layoff took place.

The remainder of this paper proceeds as follows. Section \ref{sec:literature} reviews the previous literature on the topic, and Section \ref{sec:data} discusses our data. To motivate our analysis, Section \ref{sec:edllm} presents aggregate trends in postsecondary enrollment and degree receipt, as well as the geographic variation in postsecondary enrollment. Section \ref{sec:meth} outlines our research design, while Section \ref{sec:results} presents our results. Section \ref{sec:conclusion} concludes and discusses potential directions for future work.