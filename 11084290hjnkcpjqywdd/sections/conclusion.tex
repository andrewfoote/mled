
In this paper we measure the extent to which large mass layoffs cause workers to enroll in postsecondary institutions. We contribute in four main ways to the literature on job loss and retraining. First, we focus our attention on local labor markets, while most other work has either looked at macroeconomic shocks or individual-level job loss. Second, we use a measure of local labor market shocks that is more plausibly exogenous than local unemployment rates, which previous papers have used in this context. Third, we focus on effects at the community college level, following \citet{BF1995}, while also analyzing field-specific responses. Finally, we measure how the magnitudes in responses on degree receipt correlate with labor market returns for that field.

Our results show that for the average labor market, defined as a commuting zone, an additional 100 workers laid off leads to three more first-time community college students within three years. When we compare this effect to related earlier work using a similar methodological approach, we find that educational enrollment accounts for about half of the increase in labor force non-participation following a mass layoff event. This is an optimistic finding, especially relative to recent work that shows increases in non-participation due to opioid and other drug use during hard economic times \citep{HRS2017}.

We find suggestive evidence that workers seek degrees and certificates in fields with higher labor market returns. However, the correlation between degree receipt and labor market returns are somewhat weak, which may be due to a few different mechanisms. First, some students may not know about differences in earnings potential across different majors, as has been shown often in the literature \citep{wiswall2015determinants, bakereffect}. Some students may not be able to enroll in high-return programs due to capacity constraints. Programs in health fields have high earnings returns, but also often have separate admissions requirements. There are also often waitlists for bottleneck courses across many fields. Furthermore, we only measure completion. This excludes workers who enrolled in courses and started a program, but did not complete a degree in a certain field. This latter mechanism may be important if mass layoffs push marginal students into school, or workers who have not been in school for a long time.
 
There are a number of potential directions for future research. Individual-level administrative data would allow us to follow the educational and labor market trajectories of laid off workers, as well as neighborhood-level effects. It is also important to investigate responses at for-profit colleges, which are not well-represented in our data. Recent evidence suggests that earnings outcomes at for-profits are not high \citep{DKMWP2016, DYAGK2016, CT2016, CN2015}, but students may still enroll in these institutions when faced with poor labor market conditions.

In sum, we find evidence that workers respond to mass layoffs by seeking short-duration degrees and certificates that are generally in fields with higher labor market returns. This is consistent with the idea that displaced workers seek to make new investments in specific human capital, and that there are high opportunity costs for their time. 