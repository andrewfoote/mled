\subsection{Trends in Post-Secondary Enrollment and Degree Receipt }

To better understand how college enrollment responds to macroeconomic conditions, and how college-going rates vary geographically, in this section we present enrollment trends over our study period, with a particular emphasis on the dynamics of the community college sector.



Figure \ref{fig:grenr} shows enrollment from 1996 to 2011, disaggregated by type of institution (4-year, 2-year public and 2-year private). Panel (a) of Figure \ref{fig:grenr} shows fall enrollment counts, while panel (b) shows first-time fall enrollment (i.e. newly enrolled students). In the two most recent recessions, shown as shaded bars, 2-year public enrollment increased markedly, particularly in the most recent recession. In contrast, 4-year enrollment has been secularly increasing since the beginning of our period, with no real changes in the trend in response to business cycles. Also,  while 2-year for-profit schools have received increasing scrutiny, they make up a small portion of enrollment in IPEDS.\footnote{IPEDS does not cover the for-profit sector nearly as well, as noted in Section \ref{sec:data}.}

Figure \ref{fig:grenr} also shows similar trends for degrees and certificates awarded. While there are larger enrollment counts at 4-year schools, panel (c)  shows that associate's degrees are consistently a larger share of total degree receipts, and therefore an important part of the postsecondary education sector. To get a better sense of the composition of 2-year degrees, panel (d) disaggregates the 2-year degrees into field of study, as a share of total degrees conferred. Career-technical education is a large share of total degrees, accounting for almost 60 percent of the degrees in this time period. Additionally, health has been increasing as a share of degrees, while construction and manufacturing degrees have become  less prevalent.

There are also regional differences in postsecondary and community college enrollment, as shown in the maps in Figure \ref{fig:mapenr}, which display the postsecondary enrollment by commuting zone.\footnote{As with our empirical estimates, we choose to display statistics at the commuting zone level, because many counties have no postsecondary institutions, and individuals could easily attend a school in an adjacent county.} Panel (a) of Figure \ref{fig:mapenr} shows enrollment as a share of population in the CZ, and panel (b) shows community college enrollment as a share of total enrollment.  While there is some geographic variation in enrollment in postsecondary, the variation in community college share is particularly striking. Because some areas only have access to community colleges, it is unlikely that workers who are upgrading their skills following a job loss will relocate for schooling. In fact, the strong correlation between community college enrollment and distance to the nearest institution is an often-used instrument for enrollment patterns in this literature \citep{rouse1995, XJ2013,LK2009}.

There are also significant geographic differences in field of study. Figure \ref{fig:mapawa} displays the share of overall community college awards in each field of study. While career-technical education is concentrated in the Rust Belt, there is regional variation in the particular types of programs. For example, childcare and cosmetology are concentrated in the South and in Southern California. On the other hand, health programs are much more broadly represented.

Overall, it is clear that community college enrollment is correlated with macroeconomic labor market conditions, and that career-technical education is an important piece of this response. In addition, there is a considerable amount of regional differences in educational access, enrollment, and attainment in the cross-section as well. We harness both of these levels of variation to estimate the causal effect of local economic shocks on two-year college enrollment and receipt of degrees and certificates. 



\subsection{Sample Summary Statistics}
Table \ref{tab:sumstats} displays summary statistics for the main variables we use for the analysis, for the years 1996-2013, at the commuting zone level. On average, almost 1,500 workers per year were laid off in mass layoff events in each commuting  zone. This represented around 0.6 percent of the labor force each year. In each year approximately a fifth of commuting zones had layoffs of over one percent of the labor force, while two percent of commuting zones had particularly large shocks of over five percent of the labor force laid off. 

Commuting zones have an average of between two and three community colleges, and around twice as many for-profit, two-year schools. Although there were more for-profit than public two year colleges, community colleges represented the lion's share of two-year college enrollment. 


A number of commuting zones had no community colleges or for-profit two-year colleges; these commuting zones are almost exclusively in rural areas, while in a large share of commuting zones---37 percent---two-year colleges represented the only type of postsecondary institution. Given that laid-off workers likely seek to retrain closer to home, this fact provides further motivation for focusing on the community college sector, which primarily draws students from nearby \citep{rouse1995}. 

The last rows of Table \ref{tab:sumstats} show the distribution of the content of degrees and certificates. Forty percent of awards were in career-technical fields. Of these, a quarter were in construction and manufacturing fields. An even larger share of awards were in health fields such as nursing, medical assisting, and various medical technologies such as radiology or respiratory care therapy. Only three percent of awards were in information technology. 
