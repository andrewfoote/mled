his section outlines the main data sources that we use for the analysis. While all the data we collect is reported at the county or sub-county level, we aggregate the data to the commuting zone level \citep{TS1996}. Commuting zones 
approximate the boundaries of a local economy according to commuting patterns, thus reducing the likelihood of confounding spillovers due to local migration and commuting. Moreover, because commuting zones are aggregations of counties, they are straightforward to use with publicly provided data. In our context, the use of commuting zones is useful for two other reasons. First, students are not constrained to attend community college in their county. Furthermore, many counties do not have any postsecondary institutions, meaning that we would need to drop them from the analysis, and thus lose a considerable segment of potential college enrollees. On the other hand, we believe that states are too large a definition of a local labor market, especially since two- and four-year colleges tend to draw students locally.\footnote{\citet{BL2009} find that, in Ohio, the median distance between home and four-year college was 26 miles, and over half of students lived within 50 miles.} 

\subsection{Mass Layoffs}
We use mass layoff events as our key measure of local labor market downturns, measured at the county level, which we aggregate to the commuting zone level. Between 1996 and 2013, the Bureau of Labor Statistics (BLS) compiled monthly reports on layoffs by observing the initial claims for unemployment insurance filed by workers. The BLS identified a mass layoff event when more than 50 workers filed claims against a single establishment within a five-week period. For these events, the BLS contacted the establishment to determine whether these workers experienced a layoff of at least 31 days. We use these data to quantify the size of a local labor demand shock at the county level. The data are reported by county of residence and so, for each county, we measure the number of workers residing in that county that were involved in a mass layoff in a given year. 

In other work, we have shown that mass layoffs are a viable measure of local labor market shocks \citep{FGS2015}. Most importantly, they measure a clear change in labor demand, and thus are not hampered by endogenous labor supply responses. Mass layoffs do not represent all instances of job separations for workers, only separations that resulted in 50 or more unemployment insurance claimants. Still, mass layoffs are a good measure in our setting precisely because they account for workers who lose their job, do not find work, and apply for unemployment benefits.\footnote{Unemployment benefits are usually not allowed for students. Nevertheless, unemployed workers may forgo unemployment insurance to enter training programs and community colleges.}  Thus, the number of workers involved in mass layoffs represent a group for whom job re-training is potentially the most advantageous.

In interpreting the estimates that follow, it is important to keep in mind the composition of workers who are represented by mass layoffs. \citet{Handwerker2012} look explicitly at the firms involved in mass layoffs, and note that the workers represented tend to be older and more concentrated in manufacturing and other heavily unionized industries, since this increases the likelihood that, conditional on being laid off, a worker claims unemployment benefits.

\subsection{Education Data}
Our data on enrollment and degree receipt come from the Integrated Postsecondary Education Data System (IPEDS), from the U.S. Department of Education. IPEDS data includes extensive information for all institutions of higher education that participate in federal financial aid programs, as well as some that may volunteer their own data. We focus on two measures of enrollment. These include overall and first-time fall enrollment counts, broken out by gender. 

The data in IPEDS also include information on awards, both degrees and certificates. We focus on  associate's degrees (AA and AS), and two types of certificates: one-to-four year certificates and certificates requiring less than a year.\footnote{There is also information on bachelor's degrees and graduate degrees, but these are not relevant to the margin we are studying. Bachelor's degrees at community colleges are a small and relatively new phenomenon.} We examine degrees and certificates in the aggregate and by broad field of study.\footnote{All awards are categorized by their Classification of Instructional Program (CIP) codes maintained by the U.S. Department of Education National Center for Education Statistics (NCES) and updated periodically. There are over 1,300 CIP codes. To simplify matters, we group them into broader categories. Appendix Table \ref{tab:cipapp} shows the grouping of CIP codes we use.} To further simplify, we categorize certain fields of study as ``career-technical,'' based on a system established by the California Community Colleges \citep{CCCCOtop} and used for external reporting purposes. IPEDS data also provide address data for every year for all institutions. We match each institution with its county and, thus, the number of layoffs in the local area.

Other work has discussed the drawbacks of using the IPEDS to measure the activity of for-profit sub-baccalaureate institutions \citep{Cellini2005, Cellini2010}. In particular, IPEDS tends to undercount these institutions, and is not always accurate in determining their location, which is crucial for our analysis. Nevertheless, we present results including for-profit colleges for our main analyses, with the caveat that they may be subject to  measurement error. 


\subsection{Other Data}
We supplement these main sources of data with additional information on county demographics. We use age, gender, and racial composition information from the Surveillance, Epidemiology, and End Results (SEER) program of the National Cancer Institute. We also use the SEER data to calculate the size of the working age population (ages 16-65). Additionally, we use the Local Area Unemployment Statistics (LAUS) from the BLS in order to measure the size of the labor force and the unemployment rate in the county. 